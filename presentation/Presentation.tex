\documentclass{beamer}


%------------------------------------------------------------------------
%
% Author                :   Lasercata
% Last modification     :   2024.10.21
%
%------------------------------------------------------------------------

%See https://github.com/lasercata/LaTeX_Templates for the file beamer_base.sty
\input{beamer_base_.sty}

% \usetheme{Antibes}
\usetheme{Madrid}

\title{Hack'in Trégor -- Cryptographie et stéganographie sur images}
\author{Louis} %TODO: which last name ?
\date{\today}


%---------------------------------Begin Document
\begin{document}
    \frame{\titlepage}

    \begin{frame}{Table of contents}
        \tableofcontents
    \end{frame}

    \section{Image}

    \begin{frame}{Comment voir une image ?}% {{{1
        TODO: matrice de pixels.
    \end{frame}% }}}1
    
    \section{Xor}

    \begin{frame}{Principe du \texttt{xor}}% {{{1
        \begin{block}{Table de vérité du \texttt{xor}}% {{{2
            \begin{center}
                \begin{tabular}{|c|c||c|}
                    \hline
                    $A$ & $B$ & $A \oplus B$
                    \\
                    \hline
                    \hline
                    $0$ & $0$ & $0$
                    \\
                    \hline
                    $0$ & $1$ & $1$
                    \\
                    \hline
                    $1$ & $0$ & $1$
                    \\
                    \hline
                    $1$ & $1$ & $0$
                    \\
                    \hline
                \end{tabular}
            \end{center}
        \end{block}% }}}2

        C'est le \emph{ou exclusif} : un dessert \emph{ou} l'autre, mais pas les deux !
    \end{frame}% }}}1

    \begin{frame}{Propriétés du xor}% {{{1
        \begin{block}{Table de vérité du \texttt{xor}}% {{{2
            Si
            \[
                c = m \oplus k
            \]
            alors on peut retrouver $m$ à partir de $k$ et $c$ :
            \[
                m = c \oplus k
            \]
        \end{block}% }}}2

        \vspace{12pt}
        
        Explication :

        Dans la table de vérité du \texttt{xor}, on remarque que, pour $a \in \set{0, 1}$, $\boxed{a \oplus a = 0}$ et $\boxed{0 \oplus a = a}$, d'où :
        \[
            \begin{array}{rcl}
                c \oplus k
                &=&
                (m \oplus k) \oplus k
                \\
                &=&
                m \oplus (k \oplus k)
                \\
                &=&
                m \oplus 0
                \\
                &=& m
            \end{array}
        \]
    \end{frame}% }}}1

    \begin{frame}{Xor entre entiers}% {{{1
        \begin{block}{Xor entre entiers}% {{{2
            Exemple :
            $10 \oplus 44$

            On convertit en binaire :
            \[
                \begin{array}{rcl}
                    10 &=& 1010_2
                    \\
                    44 &=& 101100_2
                \end{array}
            \]

            Puis on effectue le \texttt{xor} bit à bit :
            \[
                \begin{array}{ccccccc}
                    & 0 & 0 & 1 & 0 & 1 & 0
                    \\
                    \oplus
                    & 1 & 0 & 1 & 1 & 0 & 0
                    \\
                    \hline
                    =
                    & 1 & 0 & 0 & 1 & 1 & 0
                \end{array}
            \]

            On obtient $100110_2 = 38_{10}$.
        \end{block}% }}}2
    \end{frame}% }}}1

    \begin{frame}{Xor entre images}% {{{1
        \begin{block}{Principe}% {{{2
            On \texttt{xor} les pixels entre eux :
            \[
                \underbrace{(r_1, g_1, b_1)}_{\text{pixel image 1}}
                \oplus
                \underbrace{(r_2, g_2, b_2)}_{\text{pixel image 2}}
                =
                \underbrace{(r_1 \oplus r_2,\ g_1 \oplus g_2,\ b_1 \oplus b_2)}_{\text{pixel résultat}}
            \]
        \end{block}% }}}2

        \begin{block}{Exemple :}% {{{2
            \begin{center}
                \begin{tikzpicture}
                    % \node (1) at (0, 0) [rectangle, fill, ff4500] {\textcolor{black}{(255, 69, 0)}};
                    \node (1) at (0, 0) [rectangle, fill, ff4500] {\textcolor{black}{(255, 70, 0)}};
                    \node (xor) at (1.5, 0) {$\oplus$};
                    \node (2) at (3, 0) [rectangle, fill, 0ff] {\textcolor{black}{(0, 255, 255)}};
                    \node (eq) at (4.5, 0) {$=$};
                    \node (2) at (6.3, 0) [rectangle, fill, xor_color] {\textcolor{black}{(255, 185, 255)}};
                \end{tikzpicture}
            \end{center}
        \end{block}% }}}2

        On applique ensuite ceci pour chaque pixel.
    \end{frame}% }}}1

    \section{Stéganographie}

    \begin{frame}{Modification de pixel}% {{{1
        TODO: une petite modification du LSB des couleurs d'un pixel n'est pas visible à l'\oe il.
    \end{frame}% }}}1

    \begin{frame}{Cacher du texte dans un pixel}
        TODO
    \end{frame}

    \begin{frame}% {{{1
        \begin{center}
            \large{\bf Merci pour votre attention}
        \end{center}
    \end{frame}% }}}1
    
\end{document}
%--------------------------------------------End

% vim:foldmethod=marker:foldlevel=0
